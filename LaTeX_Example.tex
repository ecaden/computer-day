\documentclass[11pt]{article}
\usepackage{graphicx}  	% Lets you install pictures
\usepackage{amssymb}    % Cool Math symbols
\usepackage{hyperref}	% clickable hyperlinks

%Compile with 

\begin{document}

\title{Introduction to \LaTeX{}}
\author{Erica Caden}
%\date{\today}

\maketitle

% This isn't always needed
\begin{abstract}
To get the references to labels in figures, tables, and citations correct, compile the document twice. This is easiest to do with a \TeX~Integrated Development Environment, such as KILE (for Linux), TeXShop (for Macs), TeXStudio (Windows) or your favorite IDE. But you can also use any old text editor, like emacs or vim.
\end{abstract}

\section{Introduction}
As you see in Eq.~\ref{eq:simple}, we can write pretty equations.
\begin{equation}
    \label{eq:simple}
    \alpha = \frac{1}{\sqrt{\beta}}
\end{equation}
We can also write nice inline equations, $\cos (2\theta) = \cos^2 \theta - \sin^2 \theta$.

\subsection{My Favourite Subsection}
Figure~\ref{fig:myfigure} shows the simulation results.  \LaTeX~references are available online, especially~\cite{wikibooks} and~\cite{stackexchange}.
\begin{figure}[htbp]
    \centering
    \includegraphics[width=2.0in]{duck.png}
    \caption{Quackable Simulation Results}
    \label{fig:myfigure}
\end{figure}

\section{Conclusion}
See the information in Table~\ref{tab:myinfo}. We include an important fact here\footnote{And we have a footnote making a joke about that fact here}.  A printable cheat sheet is available too~\cite{cheatsheet}.  A classic dead-tree reference book is Ref.~\cite{lamport94}.

\begin{table}[htbp]
\caption{Tables of numbers}
\begin{center}
\begin{tabular}{ l | c || r } %left, centre, and right adjusted columns
  \hline                       
  1 & 2 &$ \pi$ \\
  4 & 5 & 6 \\
  7 & 8 & 9 \\  
  10&11&$\Xi$\\ \hline  
\end{tabular}
\end{center}
\label{tab:myinfo} 
\end{table}

%This is a simple bibliography 
\begin{thebibliography}{99}

\bibitem{lamport94}
  Leslie Lamport,  \emph{\LaTeX: A Document Preparation System}.  Addison Wesley, Massachusetts,  2nd Edition,  1994.
\bibitem{wikibooks} \href{https://en.wikibooks.org/wiki/LaTeX}{\texttt{https://en.wikibooks.org/wiki/LaTeX}}
\bibitem{stackexchange} \href{https://tex.stackexchange.com/}{\texttt{https://tex.stackexchange.com/}}
\bibitem{cheatsheet} \href{http://stdout.org/~winston/latex/latexsheet.pdf}{\texttt{http://stdout.org/$\sim$winston/latex/latexsheet.pdf}}

\end{thebibliography}

%This is ALWAYS the last line in your document!
\end{document}
